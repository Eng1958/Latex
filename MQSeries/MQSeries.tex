% das Papierformat zuerst
\documentclass[a4paper, 11pt]{article}

% deutsche Silbentrennung
\usepackage[ngerman]{babel}

\usepackage[utf8]{inputenc}    % wegen deutschen Umlauten

\usepackage{ulem}
\usepackage{xcolor}
\usepackage{graphicx}
\usepackage{fancyvrb}
\usepackage{lmodern}
\input{Variable.tex}

\hypersetup {
    pdftitle={ADABAS-Zugriffsmodul},
    pdfsubject={ZMOD-Library },
    pdfauthor={Dieter Engemann},
    pdfkeywords={Zugriffsmodul, ADABAS, C-Programme},
    bookmarksopen=true
}



% -----------------------------------------------------------------------------
% hier beginnt das Dokument
\begin{document}

\title{\FormatTitle{MQSeries}}
\author{Dieter Engemann\thanks{a566139.}
\\HYDRO Aluminum Rolled Products GmbH
\\Tech. IT}
% Befehl entfernen, um aktuelles Datum zu erzeugen
\date{26. August 2014}

\maketitle

\begin{figure}[!h]
  \centering
     \includegraphics[width=0.5\textwidth]{Graphics/MQSeries.png}
  \caption{MQSeries}
  \label{MQSeries}
\end{figure}


\begin{abstract}
Beschreibung des Themas.\\
\\
Version: \today{}
\end{abstract}

% Inhaltsverzeichnis anzeigen
\newpage
\tableofcontents





% Kapitel soll auf naechster Seite beginnen
\newpage
\section{Allgemeines}
\subsection{MQS Explorer}
Verzeichnis: \href{Y:/WSF-TIT/GRP/FIS/mqm/\_IBM-Install/for Win 7} 
{Y:/WSF-TIT/GRP/FIS/mqm/\_IBM-Install/for Win 7}
Duch Aufruf des Setup-Programms wird der MQS Explorer installiert.

\section{Kapitel}
\subsection{Lokaler MQSeries-Server}

IP-Name/Adresse des lokalen MQSeries-Servers: \textbf{mqs.wgb.hal.hydro.com: (149.209.104.48)}. Der 
festgelegte Port für die Kommunikation lautet: \textbf{1414}. 
\small{
\begin{verbatim}
[root@sapg]/>netstat -an | grep 149.209.104.48
tcp4       0      0  149.209.1.40.43342     149.209.104.48.1414    ESTABLISHED
tcp4       0      0  149.209.1.40.44256     149.209.104.48.1414    ESTABLISHED
tcp4       0      0  149.209.1.40.45127     149.209.104.48.1414    ESTABLISHED
tcp4       0      0  149.209.1.40.32046     149.209.104.48.1414    ESTABLISHED
\end{verbatim}}
Hier sieht man,wie die verschiedenen Prozesse auf dem SAPG Verbindung über den Port 1414
aufgebaut haben.

Auf jedem Server läuft ein Queue-Manger
\begin{Verbatim}
\item QM.MQDEGV1.1
\item QM.MQDEGV1.1
\end{Verbatim}

\subsection{Einrichten einer lokalen Queue}
Acenture legt die neuen Queues auf dem MQSeries-Server mqs und mqst an.\\

Beispiel: \textbf{LQ.DV.SAP\_PPQCP.GV.FLS}\\

Auf dem SAPG/SAPT muss ein Eintrag im Konfigurationsfile \BlueTT{\$CFGPATH/kopplung.cfg} eingetragen werden.
\begin{Verbatim} [frame=single, fontsize=\small, formatcom=\color{blue}] 
#---------------------------------------------------------------
# Konfiguration: Empfaenger
# Quelle: PP2-SAP
# Ziel:   SAPT
#---------------------------------------------------------------
PP2-SAPT:
    MQSERVER = QM.MQDEGV2.4TOCLIENT/TCP/QMMQDEGV2-4(1444)
    QueueManager = QM.MQDEGV2.4
    TargetQueueSender = AQ.DATA_SND.GV.LOGI.DV.SAP
    MQSeriesTyp = DATA_RCV
    TargetQueueReceiver-1 = LQ.DV.SAP_PPQCP.GV.FLS
    ReceiveWarteZeit = 10000
    MaxDatenLaenge = 9000
    MQSBrowse = FALSE
\end{Verbatim}

Im Dialogsystem muss eine Kopplungsdefinition und eine Tele-Kette \BlueTT{'PP2 SAPG'} bzw. \BlueTT{'PP2 SAPT'} erfasst werden. Zu der Tele-Kette gehören noch die individuellen Telegrammdefinitionen.



\newpage
\section{Websphere MQ-Protokoll}
\subsection{Schreiben in die MQSeries-Queue}
Das Schreiben in eine MQSeries-Queue wird mit dem Kommando \texttt{MQPUT} durchgeführt. Das Schreiben wird vom Queue-Manager mit \texttt{MQPUT\_REPLY} bestätigt. Anschließend wird das Schreiben mit MQCMIT bestätigt und wiederum vom Queue-Manager mit MQCMIT\_REPLY bestätigt.

\small{
\begin{verbatim}
149.209.1.40	149.209.104.48	MQ	222	MQPUT       Q=AQ.DATA_RCV.GV.LOGI.DV.SAP (1040 bytes)
149.209.104.48	149.209.1.40	MQ	630	MQPUT_REPLY Q=AQ.DATA_RCV.GV.LOGI.DV.SAP
149.209.1.40	149.209.104.48	MQ	110	MQCMIT           
149.209.104.48	149.209.1.40	MQ	110	MQCMIT_REPLY     
\end{verbatim}}

Im \texttt{MQCMIT\_REPLY} ist die Rückmeldung des Commits eingetragen.
Compl. Code \texttt{MQCC\_OK (0)} und \\
Reason Code: \texttt{MQRC\_NONE (0)}




%%%%%%%%%%%%%%%%%%%%%%%%%%%%%%%%%%%%%%%%%%%%%%%%%%%%%%%%%%%%%%%%%%%%%%%%%%%%%%%%%%%%
\section{Fehlerbehebung}
Fehlermeldungen des MQSeries-Servers (mqs und mqst) sind unter \texttt{/var/mqm/errors} und \texttt{/var/mqm/qmgrs/QM!MQDEGV1!1/errors} zu finden.
Die Fehlermeldungen des Clients (sapg und sapt) sind ebenfalls unter \texttt{/var/mqm/erros} zu finden.

\subsection{MQCONN() [(2059, X'080B') "Queue manager not available for connection"]}

Confirm that the queue manager is up and running and that the listener is running

\begin{enumerate}
	\item Ensure that the queue manager is running. You can use the \texttt{dspmq} command to verify this. The status of the queue manager should be Running.
\end{enumerate}

%%%%%%%%%%%%%%%%%%%%%%%%%%%%%%%%%%%%%%%%%%%%%%%%%%%%%%%%%%%%%%%%%%%%%%%%%%%%%%%%%
\newpage
\section{MQSeries-Kopplung}


\section{C-Programme}
\subsection{Versionen}
Die C-Programme für die MQSeries-Kopplung befinden sich unter \BlueTT{S:/sys\_6.1/64/kopplung} in
den Unterverzeichnissen der verschiedenen Versionen.\\

\small{
\begin{tabular}{lll}
\textbf{Version} & \textbf{Datum} & \textbf{Änderung} \\
\hline
MQSeriesAIX\_v2.0.20 & & \\
MQSeriesAIX\_v2.1.0 & & \\
MQSeriesAIX\_v2.1.1 & 01.09.2014 & Reihenfolge Suchpfad (HYDROLIBPATH) fuer mqm-Lib geaendert\\ 
MQSeriesAIX\_v2.1.2 & 04.11.2014 & TimeMark-Fuction mit Mikrosekunden\\ 
\hline
\end{tabular}
}


\subsection{MQSeries-API}

\paragraph {MQCONN} stellt eine Verbindung mit einem Queue-Manager mit Hilfe von Standard-Links her.
\paragraph {MQBEGIN} startet eine Arbeitseinheit, die durch den Queue-Manager koordiniert wird und
externe XA-kompatible Ressource-Manager enthalten kann. Dieses API ist mit MQSeries Version 5 eingeführt worden. Es wird für die
Koordinierung der Transaktionen , die Queues (MQPUT und MQGET unter Syncpoint-Bedingung) und Datenbank-
Updates (SQL-Kommandos) verwenden.
\paragraph {MQGET}
\paragraph {MQPUT}
\paragraph {MQPUT1} öffnet eine Queue, legt eine Message darauf ab und schließt die Queue wieder.
Dieser API-Call stellt eine Kombination von MQOPEN, MQPUT und MQCLOSE dar.
\paragraph {MQINQ} fordert Informationen über den Queue-Manager oder über eines seiner Objekte an,
wie z.B. die Anzahl der Nachrichten in einer Queue.
\paragraph {MQSET} verändert einige Attribute eines Objekts.
\paragraph {MQCMIT} gibt an, dass ein Syncpoint erreicht worden ist.
\paragraph {MQBACK} teilt dem Queue-Manager alle zurück gekommenen PUT's- und GET's-Nachrichten seit
dem letzten Syncpoint mit.
\paragraph {MQDISC} schließt die Übergabe einer Arbeitseinheit ein. Das Beenden eines Programms ohne
Unterbrechung der Verbindung zum Queue-Manager verursacht ein "Rollback" (MQBACK).




\subsection{Analyse}
Mit Hilfe des UNIX-Tools \verb|lsof| kann man sich zu einem laufenden Prozesse Informationen anzeigen lassen.
Eine MQSeries-Kopplung besteht üblicherweise aus dem Dämon und einem \verb|sender| und einem \verb|receiver|. Um alles anzuzeigen, was ein Prozesse mit einer PID geöffnet hat, kann die Option \verb|+p| verwendet werden. Dazu muss in der Prozessliste die PID des zu untersuchenden Prozesses ermittelt werden.

\tiny{
\begin{verbatim}
[root@sapg]/develop_global>lsof +p 16777394
lsof: WARNING: compiled for AIX version 6.1.3.0; this is 6.1.0.0.
COMMAND      PID USER   FD   TYPE             DEVICE  SIZE/OFF  NODE NAME
sender  16777394 sapg  cwd   VDIR              35,10       256   160 /develop (/dev/developlv)
sender  16777394 sapg    0r  VCHR                2,2       0t0 32792 /dev/null
sender  16777394 sapg    1w  VCHR                2,2       0t0 32792 /dev/null
sender  16777394 sapg    2w  VCHR                2,2       0t0 32792 /dev/null
sender  16777394 sapg    3w  VREG              10,16   1000178   190 /tmp (/dev/hd3)
sender  16777394 sapg    4r  VREG              35,10     14964    95 /develop (/dev/developlv)
sender  16777394 sapg    5w  VREG              10,16   1000178   190 /tmp (/dev/hd3)
sender  16777394 sapg    6u  IPv4 0xf1000e00002173b8 0t5169419   TCP sapg.wgb.hal.hydro.com:32257->mqs.wgb.hal.hydro.com:ibm-mqseries (ESTABLISHED)
sender  16777394 sapg   20u  IPv4 0xf1000e00006cb200       0t0   UDP *:39185
\end{verbatim}}
Hier ist zu sehen, dass der Prozess mit der PID 16777394 eine TCP-Verbindung zum MQS-Server \verb|mqs| über 
den Port (ibm\_mqseries) aufgebaut (ESTABLISHED) hat.

\tiny{
\begin{verbatim}
[root@sapg]/develop_global>lsof -i :1414
lsof: WARNING: compiled for AIX version 6.1.3.0; this is 6.1.0.0.
COMMAND       PID USER   FD   TYPE             DEVICE   SIZE/OFF NODE NAME
sender    6029420 sapg    6u  IPv4 0xf1000e00006f6bb8   0t530324  TCP sapg.wgb.hal.hydro.com:41584->mqs.wgb.hal.hydro.com:ibm-mqseries (ESTABLISHED)
sender    7340128 sapg    6u  IPv4 0xf1000e0000bbabb8  0t2967020  TCP sapg.wgb.hal.hydro.com:42221->mqs.wgb.hal.hydro.com:ibm-mqseries (ESTABLISHED)
receiver  9306214 sapg    8u  IPv4 0xf1000e0000be33b8 0t87437493  TCP sapg.wgb.hal.hydro.com:32267->mqs.wgb.hal.hydro.com:ibm-mqseries (ESTABLISHED)
sender    9633904 sapg    6u  IPv4 0xf1000e0000c04bb8  0t3681898  TCP sapg.wgb.hal.hydro.com:32266->mqs.wgb.hal.hydro.com:ibm-mqseries (ESTABLISHED)
receiver 10223788 sapg    8u  IPv4 0xf1000e0000c103b8 0t43474992  TCP sapg.wgb.hal.hydro.com:32268->mqs.wgb.hal.hydro.com:ibm-mqseries (ESTABLISHED)
sender   11403378 sapg    6u  IPv4 0xf1000e0000be83b8  0t5985084  TCP sapg.wgb.hal.hydro.com:32271->mqs.wgb.hal.hydro.com:ibm-mqseries (ESTABLISHED)
sender   11534458 sapg    6u  IPv4 0xf1000e0000c0c3b8 0t28984000  TCP sapg.wgb.hal.hydro.com:32264->mqs.wgb.hal.hydro.com:ibm-mqseries (ESTABLISHED)
sender   12058798 sapg    6u  IPv4 0xf1000e0000baebb8  0t3084306  TCP sapg.wgb.hal.hydro.com:32261->mqs.wgb.hal.hydro.com:ibm-mqseries (ESTABLISHED)
receiver 13107332 sapg    8u  IPv4 0xf1000e000021b3b8 0t37518971  TCP sapg.wgb.hal.hydro.com:32270->mqs.wgb.hal.hydro.com:ibm-mqseries (ESTABLISHED)
sender   15335672 sapg    6u  IPv4 0xf1000e0000bfd3b8   0t611125  TCP sapg.wgb.hal.hydro.com:32272->mqs.wgb.hal.hydro.com:ibm-mqseries (ESTABLISHED)
sender   16777394 sapg    6u  IPv4 0xf1000e00002173b8  0t5185871  TCP sapg.wgb.hal.hydro.com:32257->mqs.wgb.hal.hydro.com:ibm-mqseries (ESTABLISHED)
sender   16842822 sapg    6u  IPv4 0xf1000e00001ff3b8  0t5965867  TCP sapg.wgb.hal.hydro.com:32273->mqs.wgb.hal.hydro.com:ibm-mqseries (ESTABLISHED)
\end{verbatim}}
In diesem Beispiel sind alle Prozesses aufgelistet, die über die MQSeries-Port-Nr. 1414 (ibm\_mqseries) mit dem 
MQS-Server \verb|mqs| kommunizieren.





\subsection{runmqsc}
Mit dem Befehl runmqsc werden MQSC-Befehle für einen Warteschlangenmanager abgesetzt. Mithilfe von 
MQSC-Befehlen können Sie Verwaltungstasks ausführen, beispielsweise ein Objekt in einer lokalen 
Warteschlange definieren, ändern oder löschen. Die MQSC-Befehle und deren Syntax werden im Handbuch 
\href{http://www-01.ibm.com/support/knowledgecenter/api/content/nl/de/SSFKSJ_7.5.0/com.ibm.mq.ref.adm.doc/q085090_.htm}
{Referenzinformationen zu MQSC} beschrieben.
Sie müssen den Befehl runmqsc aus der Installation des Warteschlangenmanagers ausführen, mit dem Sie 
arbeiten. Um herauszufinden, welcher Installation ein Warteschlangenmanager zugeordnet ist, verwenden 
Sie den Befehl dspmq -o installation.
Beenden Sie den Befehl runmqsc mit dem Befehl end. Sie können runmqsc aber auch mit dem Befehl exit oder quit stoppen.


\section {OLD}

\begin{verbatim}
====== MQSeries ======

===== Dokumentationen =====

<WRAP><code>
"I:\Projekte\E-Books\MQSeries\MQSeries_An_Introduction_to_Messaging_and_Queuing_horaa101.pdf"
"I:\Projekte\E-Books\MQSeries\MQSeries_Application_Programming_Guide_csqzal03.pdf"
"I:\Projekte\E-Books\MQSeries\MQSeries_Application_Programming_Reference_csqzak03.pdf"
"I:\Projekte\E-Books\MQSeries\MQSeries_Application_Programming_Reference_Summary_csqzam03.pdf"

listFiles("/Projekte/E-Books/MQSeries", "[A-Z]*");
listFiles("/Projekte/E-Books/MQSeries", "Appl");
listFiles("/Projekte/E-Books/MQSeries", "[A-Z]*mp3");
listFiles("/Projekte/E-Books/MQSeries", "[A-Z]*pdf");
</code></WRAP>

===== Fehlermeldungen =====

<wrap bu>Mail-Meldungen</wrap>

  cmd "ssh sapg -l root rsh sapg -l mqm runmqsc QM.MQDEGV1.1" returned rc 255
  relevant info found
  start: 20091107.050200, stop: 20091107.050222, duration: 22 seconds, PID: 626926

bedeutet in diesem Fall, dass das ssh-Kommando nicht funktioniert hat.
\\
\\


<wrap bold>mqs [2010-04-01 12:42] mqs error on sapg-QM.MQDEGV1.1-LQ.DV.SAP.GV.ADABASDB-11187-100000-22m</wrap>
<WRAP><code>
Alarm for QM.MQDEGV1.1-LQ.DV.SAP.GV.ADABASDB on sapg, 11187 entries found, alarm limit=10000, maxDepth=100000
remaining time: 22m
--- error log for QM.MQDEGV1.1-LQ.DV.SAP.GV.ADABASDB
20100401.124208 sapg QM.MQDEGV1.1 LQ.DV.SAP.GV.ADABASDB 11187 10000 10000 100000 22m
20100401.123711 sapg QM.MQDEGV1.1 LQ.DV.SAP.GV.ADABASDB 12225 10000 10000 100000 24m
20100401.123209 sapg QM.MQDEGV1.1 LQ.DV.SAP.GV.ADABASDB 12973 10000 10000 100000 25m
20100401.122609 sapg QM.MQDEGV1.1 LQ.DV.SAP.GV.ADABASDB 11437 10000 10000 100000 22m
20100401.122207 sapg QM.MQDEGV1.1 LQ.DV.SAP.GV.ADABASDB 11903 10000 10000 100000 23m
20100401.121712 sapg QM.MQDEGV1.1 LQ.DV.SAP.GV.ADABASDB 12230 10000 10000 100000 24m
20100401.121207 sapg QM.MQDEGV1.1 LQ.DV.SAP.GV.ADABASDB 12039 10000 10000 100000 24m
20100401.120708 sapg QM.MQDEGV1.1 LQ.DV.SAP.GV.ADABASDB 12248 10000 10000 100000 24m
20100401.120208 sapg QM.MQDEGV1.1 LQ.DV.SAP.GV.ADABASDB 12464 10000 10000 100000 24m
20100401.115707 sapg QM.MQDEGV1.1 LQ.DV.SAP.GV.ADABASDB 12786 10000 10000 100000 25m
--- EOF error log for QM.MQDEGV1.1-LQ.DV.SAP.GV.ADABASDB
relevant info found
start: 20100401.124201, stop: 20100401.124220, duration: 19 seconds, PID: 766172
</code></WRAP>

Hier ist das Limit von 10000 Einträgen in der Queue überschritten worden, Telegramme werden nicht abgeholt.


===== Fehler-Auswertungen =====

im Verzeichnis <wrap bold fgblue>/var/mqm/errors</wrap> existieren Files <wrap bold fgblue>AMQ*.0.FDC</wrap>, die auf ein bestehenden Problem hinweisen. Außerdem existieren Log-Files <wrap bold fgblue>AMQERR*.LOG</wrap>.

<wrap bu>Läuft ein MQSeries-Server?</wrap>

Prinzipiell kann mit <wrap bold fgblue>telnet sapg <portnummer></wrap> (Portnummer z.B. 1444) getestet werden, ob ein MQSeries-Server antwortet.



===== Schnippsel =====

der explizite Rollback ist nach Rücksprache mit Herrn Böhm der richtige Weg, falls sie den Satz nicht in Ihrer Datenbank einfügen konnten. 
Im anderen Fall wird beim nächsten Lesezugriff ein impliziter Commit gemacht.

\end{verbatim}


% das ist wohl jetzt das Ende des Dokumentes
\end{document}


% das Papierformat zuerst
\documentclass[a4paper, 11pt]{article}

% deutsche Silbentrennung
\usepackage[ngerman]{babel}

\usepackage[utf8]{inputenc}    % wegen deutschen Umlauten

\usepackage{ulem}
\usepackage{xcolor}
\usepackage{graphicx}
\usepackage{fancyvrb}

\input{Variable.tex}

\hypersetup {
    pdftitle={Bereichsrechner - Systemumgebung},
    pdfsubject={Datenaustausch Quintiq - Oracle-DB},
    pdfauthor={Dieter Engemann},
    pdfkeywords={Quintiq, Entire Access, Oracle},
    bookmarksopen={true},
	bookmarksnumbered={true}
}

% ---------------------------------------------------------------------------------------
% hier beginnt das Dokument
\begin{document}

\title{\FormatTitle{Bereichsrechner - Systemumgebung}}
\author{Dieter Engemann\thanks{a566139.}
\\HYDRO Aluminum Rolled Products GmbH
\\Techn. IT}
% Befehl entfernen, um aktuelles Datum zu erzeugen
\date{24. September 2014}
\maketitle

\begin{figure}[!h]
  \centering
     \includegraphics[width=0.5\textwidth]{Graphics/Bereichsrechner.jpg}
  \caption{Bereichsrechner}
  \label{Bereichsrechner}
\end{figure}

\begin{abstract}
Bereichsrechner: Betriebssystem, Systemumgebung, Datenbank etc.\\
\newline
Version: \today{}

\end{abstract}


% \pagenumbering{Roman}
% Inhaltsverzeichnis anzeigen
\newpage
\tableofcontents



% *************************************************************************************
\newpage
\section{Allgemeines}
\subsection{LiramLArum}




\newpage
\section{Betriebssystem}
\subsection{Bufferpool}
\label{Bufferpool}






\newpage
\section{Shell-Umgebung}
\subsection{???}


\newpage
\section{Sofware AG - Umgebung}
\subsection{Start in eine New-Umgebung (Rechner-Upgrade}
Das einfachste, um auf einer NEW-Maschine die SAG-Produkte zu starten, ist folgendes:\\

Als User root \BlueTT{/etc/rc.sag.old} aufrufen.\\

Startet \BlueTT{Bufferpool}, \BlueTT{Adabas}, \BlueTT{Network} (ohne Verbindung nach draussen!), \BlueTT{NDV-Server}. Falls vorhanden, auch \BlueTT{nwo} und \BlueTT{tomcat}.\\

Kontrolle: \BlueTT{psg adanuc}    Ist ADABAS aktiv?


\section{Hilfsprogramme}

\begin{MyList}
\item \textbf{FPTLDIP} Anzeige der Telegramm-Pointer (TGAP09)
\item \textbf{/VAW/natural/scripts/start\_uptp} Pointer "`zusammenschieben"'
\end{MyList}

% das ist wohl jetzt das Ende des Dokumentes
\end{document}




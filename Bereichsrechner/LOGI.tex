% das Papierformat zuerst
\documentclass[a4paper, 11pt]{article}

% deutsche Silbentrennung
\usepackage[ngerman]{babel}

\usepackage[utf8]{inputenc}    % wegen deutschen Umlauten

\usepackage{ulem}
\usepackage{xcolor}
\usepackage{graphicx}
\usepackage{fancyvrb}

\input{Variable.tex}

\hypersetup {
    pdftitle={Bereichsrechner - Systemumgebung},
    pdfsubject={Datenaustausch Quintiq - Oracle-DB},
    pdfauthor={Dieter Engemann},
    pdfkeywords={Quintiq, Entire Access, Oracle},
    bookmarksopen={true},
	bookmarksnumbered={true}
}

%\setlength{\parindent}{4em}
\setlength{\parskip}{0.5em}
% \renewcommand{\baselinestretch}{2.0}

% ---------------------------------------------------------------------------------------
% hier beginnt das Dokument
\begin{document}

\title{\FormatTitle{Bereichsrechner - LOGI}}
\author{Dieter Engemann\thanks{a566139.}
\\HYDRO Aluminum Rolled Products GmbH
\\Techn. IT}
% Befehl entfernen, um aktuelles Datum zu erzeugen
\date{17. November 2014}
\maketitle

\begin{figure}[!h]
  \centering
     \includegraphics[width=0.5\textwidth]{Graphics/Logistik.jpg}
  \caption{LOGI}
  \label{LOGI}
\end{figure}

\begin{abstract}
Bereichsrechner: Logistik, Torwaage etc.

Version: \today{}

\end{abstract}


% \pagenumbering{Roman}
% Inhaltsverzeichnis anzeigen
\newpage
\tableofcontents



% *************************************************************************************
\newpage
\section{Allgemeines}
\subsection{LiramLArum}

This is the first paragraph, contains some text to test the paragraph
interlining, paragraph indentation and some other features. Also, is 
easy to see how new paragraphs are defined by simply entering a double 
blank space.
 
Hello,  here  is  some  text  without  a  meaning.   This  text  should
show what a printed text will look like at this... 



\begin{Verbatim}

====== LOGI ======




===== Packstücke =====
Sollen in LOGI Packstücke mit [[MIF Material]] gelöscht werden ([[LOGI-Packstücke|Packstücke]] sind in [[SAP]] nicht bekannt !!) muss darauf geachtet werden, dass die Rollen in SAP von den BAV auch bearbeitet (gelöscht oder so) werden. 

Ansonsten tauchen sie in der Differenzliste wieder auf. 

===== LOGI - LOHN =====

  *Ein SOFN-Telegramm (//"Schranke Oeffnen"//) wird beim Empfang eines KART-Telegrammes (//"Karte gelesen"//) vom Bereichsrechner [[LOHN|LOHN]] an den Bereichsrechner LOHN zurückgesendet.
  *Aus dem Subprogram **TWN-WIEG[LOSTEP]]** wird ein SOFN-Telegramm an den Bereichsrechner LOHN gesendet.


==== Lager ====


==== Stanzomat ====

<file>
laut meinen Unterlagen sollte der Stanzomat z.Z. folgendermaßen 
eingestellt sein:

Baudrate:	2400 Bd
Parität:		gerade		bei Nichtfunktionieren evt. ungerade ausprobieren, "un" ist bei mir handschriftlich durchgestrichen
Datenbits	7
Stopbits	1

Buchse:	25pol. female

Pinbelegung:	Standard RS232-C
</file>

Ansonsten ist die Schnittstelle über Jumper auf der Platine einstellbar. Wenn nichts funktionieren sollte, muß der Stanzomat geöffnet werden.


===== Torwaage =====

Bei der Verwiegung wird ein WIEG-Ereignis an das Produktionsbuch gesendet. \\
Das Ereignis setzt sich zusammen aus
Ereigins-Art: **WIEG**\\
ID: **LK**\\
Kennzeichen: **DAUR-SN 97**\\


  *[[Coil-Aussenlager]]
  *[[PBZ-Torwaage-Auswertungen]]
===== Projekte =====

  *[[Ablösung KVO-Dialog]] 
  *[[LOGI-Platzdatei]]
  *[[Focus Beverage GV]] 





Ladelisten
Einmal am Tag wird überprüft (5:20 Uhr; Programm: LOPBDEL2[LOSTEP]), welche Ladelisten den Status **5** haben. Bei dieser Überprüfung werden alle Packstücke, die zu dieser Ladeliste gehören, gelöscht. Das Löschen wird im Logfile /daten/delete_ladelisten.sh.<Datum-Uhrzeit>.liste protokolliert.

In diesem Programm werden ab sofort (03.07.2009) auch die [[PRLA-Telegramme]] im Telestack gelöscht.


Ausdruck Etikett mit Ladelisten-Nr
Wenn das erste Packstück der Ladeliste ausgelagert wird, soll ein Etikett über den Labelserver mit der Ladelisten-Nr ausgedruckt werden.

Es muss ein ETIK-Telegramm mit der folgenden Struktur aufgebaut werden:
<code>
| Formular-Druck                                                      Seite: 1 |
| -----------------------------------------------------------------------------|
|    Drucker prt818____    Kopien 1__              Datum / Zeit 20091118 011228|
|                                                                              |
| 01 ##E    #logistik/etiketten/LALI___________________________________________|
| 02 ##P0001#139_______________________________________________________________|
| 03 ##P0002#158997____________________________________________________________|


Um das Etikett als PDF-File in einer E-Mail zu empfangen, muss der Drucker prt836 und die E-Mail Adresse angegeben werden.

| Formular-Druck                                                      Seite: 1 |
| -----------------------------------------------------------------------------|
|    Drucker prt836____    Kopien 1__              Datum / Zeit 20091119 095514|
|                                                                              |
| 01 ##E    #logistik/etiketten/LALI___________________________________________|
| 02 ##P0002#159129____________________________________________________________|
| 03 ##N    #hans-dieter.engemann@hydro.com+++ETI-LALI_________________________|


</code>

===== Lagerhochhaus L1 =====
==== I-Punkte IP1/IP2 ====
Der Transport an den I-Punktemn bei der Einlagerung soll langsam sein (nicht immer)

Einlagerung I-Punkt: LOPDIP-1[LOIP]/LOPDIP-2[LOIP]

Programme schreiben Fehlerprotokolle

  *[[LH1-Einlagerung Packereien]]
===== Lagerhochhaus L2 =====

<wrap bu>Platz- und Anlagenübersicht der Auslagerstrecke im Lagerhochhaus 2</wrap>

{|class="inline" 
|-
! Platz
! Bezeichnung
! Anlage
|- 
|
|
|
|} 


  *[[LH2-Umbau-RFZ6-RFZ7]]
  *[[LH2-Gewichtsanpassungen]] Auflastung Gasse 7
===== Kartenleser Torwaage =====
Die beiden Torwaage-Kartenleser TWKA und TWKA werden beide über den Terminalserver twts16a angesprochen.

<wrap bu>Überwachung durch das Status-Telegramm</wrap>

Über die ZLTK-Schnittstelle werden regelmäßig Status-Anforderunge an die einzelnen Kartenleser gesendet und die Rückantwort an die ZLTK-Schnittstelle zurückgesendet. Das gleiche Verfahren ist schon beim Rechner [[Lohn]] für die Besucherkartenleser realisiert.

{| class="inline"
!Quelle
!Ziel
!Programm
|-
|ZLTK
|LOGI
|AWNTKLST
|-
|TWKA/TWKE
|LOGI
|AWNTKLST
|}

<wrap bu>Verhalten des Kartenlesers ohne Rechner-Kopplung</wrap>

Bei dem Fall, dass eine Kartenleser-Kopplung bzw. der Rechner LOGI nicht vorhanden ist, verhält sich der Kartenleser folgendermassen:

<WRAP 50%>
  *Wenn keine Quittung für das gesendete Telegramm empfangen wird, erscheint dieMeldung <wrap fgblue>**Bitte warten**</wrap>.
  *Es wird versucht, das Datentelegramm dreimal zu senden.
  *Wenn nach dem dritten Mal keine Quittung empfangen worden ist, erscheint die Meldung  <wrap fgblue>**Störung, bitte melden**</wrap>.
</WRAP>
Die Telegramme werden nicht gepuffert, auch nach Wiederherstellung der Rechnerverbindung werden keine Telegramme gesendet.

==== Neue Karte erfassen ====

Eine neue Karte für einen LKW wird mit Absprache der Personalabteilung (Heinz-Josef Pelzer, Tel.: 1273) erfasst. Die Personalabteilung führt eine Liste mit vergebenen Karten-Nr. und stellt neue Karten zur Verfügung.

Auf dem LOGI-Rechner muss die Karte im Dialog **"Karten-LKW-Liste"** entsprechend der Vorgaben erfasst werden.
<WRAP 60%><code>
DI ___   LOGI   KLL Karten-LKW-Liste                                   08.02.11 
FU ___ DR ___   Dialogmodus: Ändern                                    07:49:02 
                                                                                
 Karten-ID:    1666                                                             
                                                                                
 LKW-Daten                                                                      
                                                                                
 Nationalitaet D                       Zuladen-KNZ                              
                                                                                
 Kennzeichen   NE-EV 1                 Baenderfahrzeug X                        
                                                                                
 Spedition     KLEFISCH                Leergewicht     14000                    
                                                                                
                                                                                
 Bemerkung                                                                      
BAENDER                                                                         
</code></WRAP>                                                                                 
  *Programmänderungen müssen abhängig von der Funktion des LKWs erfolgen, in der Regel aber nicht.
  *Eine Erfassung der Karte auf dem [[LOHN]]-Rechner muss ebenfalls erfolgen.
======  ======



</pre>



\end{Verbatim}

\subsection{Etikettenausdruck}

Es soll für den Nummernkreis 4xxxxx (Ladeliste) folgendes Etikett automatisch bei der Auslagerung erzeugt werden. 

Etikett mit Text "gesperrt, Zurück in den Produktionsbetrieb"

Für den Nummerkreis 6xxxxx (Ladeliste) soll folgendes Etikett automatisch am Auslagerplatz gedruckt werden.

Etikett mit Text "Auslagerung für Färberei"

Sie dazu auch Dialog "702 Ladelisten auslagern" (Gruppe VER3)

Es betrifft die Drucker-Nr. (Logi) 818, 819 und 820.

Ausdruck erfolgt auf den Rampenplätzen wie bei den Versandetiketten.

hier der Pfad zum Etikett:
/logistik/etiketten/

Der Name des Etiketts: 
UNI1 

Der vollständige Pfad:
/logistik/etiketten/UNI1 

UNI - Abkürzung für "universell".

\subsection{Etikettenausdruck Versandhalle LH1 (Rampen)}
Die Etiketten werden bei einem Platzzugang einer Palette auf den Rampen-Plätzen VA414, VA424, VA444 
und VA464 gedruckt. Damit die Etiketten eher gedruckt werden, soll der Ausdruck schon bei einem 
Platzzugang auf den Platz \textbf{VA407} (Verteilwagen) angestoßen werden.

\begin{Verbatim}
             +--------+ +--------+ +--------+ +--------+ +--------+
             | APL5   | | APL4   | | APL3   | | APL2   | | APL1   |
             +--------+ +--------+ +--------+ +--------+ +--------+
[6]          | VA452  | | VA442| | | VA432  | | VA422  | | VA412  |
+--------+   +--------+ +--------+ +--------+ +--------+ +--------+
| VA464  |   | VA453  | | VA443| | | VA433  | | VA423  | | VA413  |
+--------+   +--------+ +--------+ +--------+ +--------+ +--------+
                                  
                                   +--------+
                                   | VA407  |
                                   +--------+
                                   
+--------+                         +--------+            +--------+
| VA444  |                         | VA424  |            | VA414  |
+--------+                         +--------+            +--------+
| VA445  |                         | VA425  |            | VA415  |
+--------+[4]                   [2]+--------+         [1]+--------+

\end{Verbatim}

\subsection{Etikettenausdruck beim Starten der Ladeliste}
Beim Start der Ladeliste wird ein Etikett mit dem Kürzel \textbf{139...} und der Ladelisten-Nr. gedruckt.

\begin{Verbatim}
| Formular-Druck                                                      Seite: 1 |
| -----------------------------------------------------------------------------|
|    Drucker prt818____    Kopien 1__              Datum / Zeit 20130326 063944|
|                                                                              |
| 01 ##E    #logistik/etiketten/LALI___________________________________________|
| 02 ##P0001#139......_________________________________________________________|
| 03 ##P0002#285428____________________________________________________________|
| 04 ##N    #hans-dieter.engemann@hydro.com+++ETI-LALI_________________________|
\end{Verbatim}

Abhängig von den Packstücken in der Ladeliste soll das Etikett auf einem Drucker in der Versandhalle 
LH1 oder LH2 aoder bei einem Mix von Packstücken auf den Druckern in beiden Versandhallen gedruckt 
werden.

\begin{tabular} {ll}
\textbf{Bereich} & \textbf{Drucker} \\
\hline
Lagerhochhaus LH1 &  818 \\    
Lagerhochhaus LH2 &  820 \\  
\end{tabular}



\end{document}
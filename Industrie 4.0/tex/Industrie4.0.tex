% das Papierformat zuerst
\documentclass[a4paper, 10pt]{article}

% deutsche Silbentrennung
\usepackage[ngerman]{babel}

\usepackage[utf8]{inputenc}    % wegen deutschen Umlauten

\usepackage{ulem}
\usepackage{xcolor}
\usepackage{graphicx}
\usepackage{fancyvrb}

\input{Variable.tex}

\hypersetup {
    pdftitle={Quintiq-Kopplung},
    pdfsubject={BANF, Bestellungen, Kostenstellen},
    pdfauthor={Dieter Engemann},
    pdfkeywords={BANF, SAP},
    bookmarksopen=true
}

\makeindex

%
% this makes list spacing much better.
%
%\newenvironment{MyList}{
%\begin{itemize}
%  \setlength{\itemsep}{1pt}
%  \setlength{\parskip}{0pt}
%  \setlength{\parsep}{0pt}}{\end{itemize}
%}

% ---------------------------------------------------------------------------------------
% hier beginnt das Dokument
\begin{document}

\title{\FormatTitle{Industrie 4.0}}
\author{Dieter Engemann\thanks{a566139.}
\\HYDRO Aluminum Rolled Products GmbH
\\Tech. IT}
% Befehl entfernen, um aktuelles Datum zu erzeugen
\date{14. November 2014}

\maketitle

\begin{figure}[!h]
  \centering
     \includegraphics[width=0.7\textwidth]{../Graphics/Industrie40.png}
  \caption{Industrie 4.0}
  \label{Industrie4.0}
\end{figure}

\begin{abstract}
etc.\\
Version: \today{}
\end{abstract}

% Inhaltsverzeichnis anzeigen
\newpage
\tableofcontents





% Kapitel soll auf naechster Seite beginnen
\newpage

\section{Industrie 4.0}

\subsection{Brainstorming}
\subsubsection{}
\paragraph{Grundlage ist:} Verfgbarkeit von aktuellen nachvollziehbaren Informationen an jedem Ort
und zu jeder Zeit der Produktion/Prozesse.\\
\begin{itemize}
  \item Selbst-Organisation
  \item Selbst-Optimierung
  \item Selbst-Diagnose
  \item Selbst-Konfiguration
\end{itemize}
=> größere Dynamik 


\paragraph{Vernetzung} aller beteiligten Objekte, Informationen, Ressourcen etc.
\paragraph{Unternehmensübergreifende Vernetzung} mit resultierender intensiverer Koopeartion
zwischen Lieferant / Produzent / Kunde.

\paragraph{Zugriff der Automatisierungebenen} auf einen gemeinsamen Datenbestand (Cloud) um
Emtscheidungen treffen zu können.
\begin{itemize}
  \item Feldebene
  \item Steuerung
  \item Scada / Visualisierung
  \item MES /FLS /Shop Floor
  \item ERP /SAP
\end{itemize}

\paragraph{Security} / Datensicherheit / Verschlüsselung

\paragraph{Standardisierung} der Prozessabläufe, um eine höhere Flexibilität und Dynamik
anbieten zu können. Individualisierung von Kundenwünschen und damit einhergehende Anpassungen
der Produktionsschritte.

\paragraph{Einheitliche Protokolle:} Unterstützung von Standard-Protokollen
\begin{itemize}
  \item OPC-Protokoll als M2M-Protokoll
  \item Odette File Transfer Protkoll (Automobil) 
\end{itemize}

\paragraph{Anbindung vielfältiger Objekte:} Mobilfunk-Geräte, Tablets, Messgeräte, Sensoren

\newpage
\section{Hydro}
Ansätze vorhanden:


plus 
\begin{itemize}
  \item Anbindungen vieler Maschinen vorhanden
  \item Daten-/Materialfluss und Datenkontrolle vorhanden 
  \item Rel. große Vernetzung (Kabel)
 
\end{itemize}


minus
\begin{itemize}
  \item kein gemeinsamer Datenpool
  \item Geringe Vernetzung auf Funk-Ebene für mobile Geräte etc
  \item Keine durchgängiger Standards auf MES-Ebene (Dialog-Ebene)
  \item Keine Industriestandard-Protokolle
\end{itemize}


\end{document}
